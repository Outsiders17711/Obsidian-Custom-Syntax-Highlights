\documentclass{article}

% Packages
\usepackage[utf8]{inputenc}
\usepackage{amsmath}
\usepackage{graphicx}
\usepackage{hyperref}

% Title and Author
\title{Custom Syntax Highlights Test}
\author{Outsiders17711}
\date{\today}

\begin{document}

% Title Page
\maketitle

% Abstract
\begin{abstract}
  This document serves as a comprehensive test file for LaTeX syntax highlighting in Obsidian. It demonstrates various LaTeX commands, environments, and structures to validate proper syntax colouring and formatting.
\end{abstract}

% Section
\section{Plugin Features}
This plugin provides syntax highlighting for custom file extensions in Obsidian. Key features include inline math $\alpha + \beta = \gamma$ and references to Figure \ref{fig:plugin-demo}.

% Subsection
\subsection{Supported Extensions}
The plugin supports various file types:
\begin{itemize}
  \item \texttt{.tex} files for LaTeX documents
  \item \texttt{.json} files for configuration
  \item \texttt{.bib} files for bibliography management
  \item \texttt{.py} files for Python scripts
  \item \texttt{.js} files for JavaScript code
  \item \textttt{.yaml} files for YAML configurations
\end{itemize}

% Equation
\section{Configuration Example}
The plugin configuration can be expressed mathematically:
\begin{equation}
  \text{Highlighting Quality} = f(\text{Extension Support}, \text{Syntax Rules})
\end{equation}

% Figure
\section{Plugin Demonstration}
Here is a visual representation of the plugin functionality:
\begin{figure}[h!]
  \centering
  \includegraphics[width=0.5\textwidth]{syntax-highlight-demo}
  \caption{Demonstration of syntax highlighting capabilities.}
  \label{fig:plugin-demo}
\end{figure}

% Conclusion
\section{Conclusion}
This test document validates the LaTeX syntax highlighting functionality of the plugin, ensuring proper rendering and code recognition.

% Bibliography
\begin{thebibliography}{9}
  \bibitem{obsidian2021}
  Obsidian Team, \textit{Obsidian: A Knowledge Base That Works on Local Markdown Files}, 2021.

  \bibitem{latexproject}
  LaTeX Project Team, \textit{LaTeX Documentation and Guides}, \url{https://www.latex-project.org/}, 2024.
\end{thebibliography}

\end{document}
